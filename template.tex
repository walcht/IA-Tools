% $Id: template.tex 11 2007-04-03 22:25:53Z jpeltier $
\documentclass{vgtc}                          % final (conference style)

\ifpdf%                                % if we use pdflatex
  \pdfoutput=1\relax                   % create PDFs from pdfLaTeX
  \pdfcompresslevel=9                  % PDF Compression
  \pdfoptionpdfminorversion=7          % create PDF 1.7
  \ExecuteOptions{pdftex}
  \usepackage{graphicx}                % allow us to embed graphics files
  \DeclareGraphicsExtensions{.pdf,.png,.jpg,.jpeg} % for pdflatex we expect .pdf, .png, or .jpg files
\else%                                 % else we use pure latex
  \ExecuteOptions{dvips}
  \usepackage{graphicx}                % allow us to embed graphics files
  \DeclareGraphicsExtensions{.eps}     % for pure latex we expect eps files
\fi%

%% it is recomended to use ``\autoref{sec:bla}'' instead of ``Fig.~\ref{sec:bla}''
\graphicspath{{figures/}{pictures/}{images/}{./}} % where to search for the images

\usepackage{microtype}                 % use micro-typography (slightly more compact, better to read)
\PassOptionsToPackage{warn}{textcomp}  % to address font issues with \textrightarrow
\usepackage{textcomp}                  % use better special symbols
\usepackage{mathptmx}                  % use matching math font
\usepackage{times}                     % we use Times as the main font
\renewcommand*\ttdefault{txtt}         % a nicer typewriter font
\usepackage{tabu}                      % only used for the table example
\usepackage{booktabs}                  % only used for the table 
\usepackage{url}
\usepackage{tabularray}
\usepackage[backend=biber]{biblatex}
\addbibresource{template.bib}

\onlineid{0}
\vgtccategory{Research}

\title{The state-of-art
platforms and tools which can be used to develop immersive analytical
applications}

\author{Walid Chtioui\thanks{e-mail: walid.chtioui@ensi-uma.tn}\\ %
        \scriptsize Computer Science M.Sc. Student %
\and Ed Grimley\thanks{e-mail: <EMANIL HERE ACHREF!>}\\ %
     \scriptsize Computer Science M.Sc. Student}

%% A teaser figure can be included as follows, but is not recommended since
%% the space is now taken up by a full width abstract.
%\teaser{
%  \includegraphics[width=1.5in]{sample.eps}
%  \caption{Lookit! Lookit!}
%}

\abstract{This paper presents an overview of the latest state-of-the-art
platforms and tools which can be used to develop immersive analytics
(hereafter IA) applications. The term platform refers to programs that provide
a high-level application programming interface (API) along side a set
of extensive XR features which allow developers to create complex XR
experiences. Initially, an overview of what we believe are the major extended
reality (hereafter XR) development platforms is provided after which a table
of, what we think are, important-to-have features and by which platforms they
are supported is presented.
} % end of abstract

\CCScatlist{ 
    {HEY}{HEY}
}

%%%%%%%%%%%%%%%%%%%%%%%%%%%%%%%%%%%%%%%%%%%%%%%%%%%%%%%%%%%%%%%%
%%%%%%%%%%%%%%%%%%%%%% START OF THE PAPER %%%%%%%%%%%%%%%%%%%%%%
%%%%%%%%%%%%%%%%%%%%%%%%%%%%%%%%%%%%%%%%%%%%%%%%%%%%%%%%%%%%%%%%%

\begin{document}
\firstsection{Introduction}
\maketitle
This template is for papers of VGTC-sponsored conferences which are
\emph{\textbf{not}} published in a special issue of TVCG.

\section{Platforms}
\subsection{Unity}
Unity has a wide adoption in the world of XR thanks to its unified workflow
and support for various XR platforms - build once, run
everywhere -. Unity supports an extensive set of XR vendor-specific software
development kits (hereafter SDK) including: Apple's ARKit, Google's ARCore,
Microsoft's HoloLens and OpenXR. Following the announcement of Apple's
mixed reality (MR) headset Vision Pro in Apple's Worldwide Developers
Conference (WWDC) 2023, Unity was announced to provide native support for
Vision's Pro operating system VisionOS \cite{web:vision_pro_unity}.
Unity also provides a set of XR packages that are built on top of these vendor
plugins to add application-level development tools \cite{unity:xr_packages}.
For instance, AR Foundation is an industry-standard framework that provides
support for various AR features such as: object tracking and plane detection.
\subsection{Unreal Engine}
\subsection{Comparison}
Figure \ref{table:1} provides a comparison between the previously discussed
platforms in terms of support for vendor-specific SDKs and a set of features.

\begin{table}[h!]
	\centering

	\begin{tabular}{l c c}
		\toprule
		                  & \multicolumn{2}{c}{\textbf{Platform}}                 \\
		\cmidrule(l){2-3}
		\textbf{Strain}   & Unity                                 & Unreal Engine \\
		\midrule
		ARCore            & X                                     & X             \\
		ARKit             & X                                     & X             \\
		Magic Leap        & X                                     &               \\
		Microsof HoloLens & X                                     &               \\
		OpenXR            & X                                     & X             \\
		Oculus            & X                                     & X             \\
		WebXR             &                                       &               \\
		VisionOS          & X                                     &               \\

		\bottomrule
	\end{tabular}


	\medskip

	\caption{Per-platform supported SDKs and XR features.}
	\label{table:1}
\end{table}

\section{Toolkits}

\subsection{DXR Toolkit}

Sicat et al. proposed DXR \cite{dxr_toolkit}; an IA toolkit built on top of the Unity game engine
that allows fast prototyping and iteration for non-experienced users; i.e.
users with no or little programming knowledge in XR and Unity.
Alongside the data input, DXR takes a specification file written in JavaScript
Object Notation (hereafter JSON) from which visualisations are created.
The specification file is described in Vega-Lite declarative grammar \cite{vega_lite}
(only what should be achieved has to be provided, not how) making it suitable
for users with no programming experience to rapidly realise immersive
visualisations. This file can be edited in a separate text editor or through
the use of a GUI with pre-configured set of parameters. DXR also provides
built-in specification templates for common visualisations such as: Scatter
plots and bar charts. This extends the scope of users even more to include
those without any technical experience. Although the authors claim that DXR
provides suitable flexibility, the scope of that flexibility seems to be
limited, among other things, to providing custom graphical markers, custom
encoding channels - a visualization channel is a visualisation parameter affected by some
data attribute(s), such as object color affected by temperature data attribute in some dataset -
and other visualisation-type specific properties. That limits users to a
templated and common set of visualisations such as scatter plots, bar charts
and radial bars. There is also no mention of real-time data support thus
limiting the use case of DXR to offline data only.

\smallskip

\noindent As the authors have explicitly mentioned, DXR is meant for
prototyping and exploring designs, it is not designed to handle visualisations
of large datasets. On HoloLens, for datasets with more than approximately
a thousand item, suboptimal - less that 60 frames per second (hereafter FPS) -
performance has been observed. Nonetheless, the authors argue that DXR can still
be useful for quickly and cheaply prototyping large dataset designs before
moving to specialized, optimized and detailed implementations.

\section{Prototypes}
\subsection{Uplift}
Barrett et al. proposed Uplift \cite{uplift_prototype}; an in-place
collaborative visual analytics prototype targeting users with diverse
expertise in the domain of microgrids. Uplift is designed for casual visual
analysis use-cases; i.e. to be used to easily identify, in a relatively short
time, key patterns in complex visualized data. The requirements for the
prototype were initially provided and subsequently modified, through multiple
feedback sessions, by a wide range of stakeholders including microgrid project
and energy systems experts.

\smallskip

\noindent Uplift relies more than just AR headsets to bring casual collaborative visual
analytics to a multitude of microgrid stakeholders. A tabletop display showing
a geographical map of the campus grid is used as a central platform where users
are supposed to gather around and interact with widgets placed on top of it.
Uplift also makes use of tangible
widgets which are physical and interactive elements that control visualization
parameters (for instance by affecting sliders). The prototype also relies on
scaled-down physical models of buildings that are translucent which allows
the color of the surface on which they are placed to be used as an appealing
visualization channel. AR is used to display multiple 3D data types on top of
the tabletop and 2D graphs alongside legends around it. On top of these, Uplift
uses a large display to either replicate the content of the tabletop or show
additional visualizations.

\smallskip

\noindent Through the feedback of 16 participants who tried the prototype, Uplift was
proven to be potentially useful for microgrid-related data analytics.

\smallskip

\noindent Although Uplift was designed for microgrid-related systems,
the authors claim that its applicability domain can be extended to include
other domains that rely on analysis of complex spatial data such as
the construction industry. However, the use of a wide range of technologies
and gadgets makes Uplift a specialized solution that we believe isn't yet
ready for wide deployment. For instance, on top of using Vuforia for tabletop
tracking, Uplift uses an extra proprietary tracking software with four cameras
to track the tangible widgets. Such tracking could have instead been done in
Vuforia therefore removing the need to add cameras to the scene and making the
solution much more self-contained. This is especially true since Vuforia provides
an official plugin for Unity development \cite{unity:vuforia_plugin}.
Moreover, although the topic of real-time monitoring of the microgrid was seen
as beneficial for operators by expert stakeholders, Uplift didn't provide any
solutions to tackle such use case.

\section{Collaboration Tools and Techniques}
\section{Interaction Tools and Techniques}
\section{Navigation Tools and Techniques}

\section{Conclusion}
TO BE WRITTEN AT THE VERY END

\acknowledgments{TO BE WRITTEN AT THE VERY END}
\printbibliography
\end{document}
